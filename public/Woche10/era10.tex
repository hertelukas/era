%%
% This is an Overleaf template for presentations
% using the TUM Corporate Desing https://www.tum.de/cd
%
% For further details on how to use the template, take a look at our
% GitLab repository and browse through our test documents
% https://gitlab.lrz.de/latex4ei/tum-templates.
%
% The tumbeamer class is based on the beamer class.
% If you need further customization please consult the beamer class guide
% https://ctan.org/pkg/beamer.
% Additional class options are passed down to the base class.
%
% If you encounter any bugs or undesired behaviour, please raise an issue
% in our GitLab repository
% https://gitlab.lrz.de/latex4ei/tum-templates/issues
% and provide a description and minimal working example of your problem.
%%


\documentclass[
  german,            % define the document language (english, german)
  aspectratio=169,    % define the aspect ratio (169, 43)
  % handout=2on1,       % create handout with multiple slides (2on1, 4on1)
  % partpage=false,     % insert page at beginning of parts (true, false)
  sectionpage=false,   % insert page at beginning of sections (true, false)
]{tumbeamer}


% load additional packages
\usepackage{booktabs}
\usepackage{listings}
\usepackage{tikz-timing}

% presentation metadata
\title{Einführung in die Rechnerarchitektur}
\subtitle{Schaltwerke \& VHDL}
\author{Lukas Hertel}

\institute{\theChairName\\\theDepartmentName\\\theUniversityName}
\date[10.01.2022]{10. Januar 2022}

\footline{\insertauthor~|~\insertshorttitle~|~\insertshortdate}


% macro to configure the style of the presentation
\TUMbeamersetup{
  title page = TUM tower,         % style of the title page
  part page = TUM toc,            % style of part pages
  section page = TUM toc,         % style of section pages
  content page = TUM more space,  % style of normal content pages
  tower scale = 1.0,              % scaling factor of TUM tower (if used)
  headline = TUM threeliner,      % which variation of headline to use
  footline = TUM default,         % which variation of footline to use
  % configure on which pages headlines and footlines should be printed
  headline on = {title page},
  footline on = {every page, title page=false},
}

% available frame styles for title page, part page, and section page:
% TUM default, TUM tower, TUM centered,
% TUM blue default, TUM blue tower, TUM blue centered,
% TUM shaded default, TUM shaded tower, TUM shaded centered,
% TUM flags
%
% additional frame styles for part page and section page:
% TUM toc
%
% available frame styles for content pages:
% TUM default, TUM more space
%
% available headline options:
% TUM empty, TUM oneliner, TUM twoliner, TUM threeliner, TUM logothreeliner
%
% available footline options:
% TUM empty, TUM default, TUM infoline


\begin{document}

\maketitle
\begin{frame}{Organisatorisches}
	\begin{itemize}
		\item Prüfungsanmeldung bis zum 15.01.2022
	\end{itemize}
\end{frame}

\section{Schaltwerke}
\begin{frame}{Schaltwerke}
	\begin{itemize}
		\item Schaltnetze mit Gedächtnis
		\item Speicherung möglich durch Rückkopplung
		\item ``Latch'' ist nicht getaktet
		\item ``Flip-Flop'' Pegel- oder Flankengesteuert (Getaktet)
	\end{itemize}
\end{frame}
\begin{frame}{RS-Latch}
\centering
\begin{tabular}[t]{cc|c}
	\(S\)    & \(R\) & \(Q\) \\
	\hline
	0         & 0     &  \\
	0        & 1     & \\
	1        & 0     & \\
	1         & 1     &  \\
\end{tabular}
\end{frame}

\begin{frame}{Pegel- und Flankensteuerung}
	\centering
	\renewcommand{\arraystretch}{2.5}
	\setlength{\tabcolsep}{20pt}
	\begin{tabular}{|c|c|c|}
		\hline
		& Änderung wenn... & Beispiel \\
		\hline
		Pegelsteuerung &  &  \\
		\hline
		Flankensteuerung &  &  \\
		\hline
	\end{tabular}
\end{frame}

\begin{frame}{Signalverläufe}
	\begin{tikztimingtable}[xscale=2,yscale=1.5]
		CLK                 & 19{C} \\
		S                    & LL  lHl LL  lHh hLl Llh hLl LL  lHl L \\
		R                   & Hhl LL  lhL LL  Llh hLl lHh hLl LL  L \\
		RS-Latch               & \\
		getaktetes RS-FF    & \\
		MS-D-FF                & \\
		\extracode
		
		\begin{background}[help lines]
			\vertlines[dotted]{1,3,...,17}
			\vertlines[dashed]{2,4,...,18}
			\horlines{}
		\end{background}
	\end{tikztimingtable}
	
\end{frame}
\section{Flip-Flop Arten}
\begin{frame}{Nicht getaktetes RS-Latch}
\end{frame}
\begin{frame}{Taktpegelgesteuertes RS-Flip-Flop}
\end{frame}
\begin{frame}{D-Flip-Flop}
		\centering
	\begin{tabular}[t]{c|c}
		\(D\)  & \(Q\) \\
		\hline
		0         & \\
		1         & \\
	\end{tabular}
\end{frame}
\begin{frame}{Taktpegelgesteuertes D-Flip-Flop}
\end{frame}
\begin{frame}{Taktflankengesteuertes D-Flip-Flop}
	\begin{itemize}
		\item Wie verändern um auf steigende Flanken zu reagieren?
	\end{itemize}
\end{frame}
\begin{frame}{JK-Flip-Flop Tabelle}
	\centering
	\begin{tabular}[t]{cc|c}
		\(J\)    & \(K\) & \(Q\) \\
		\hline
		0         & 0     &  \\
		0        & 1     & \\
		1        & 0     & \\
		1         & 1     &  \\
	\end{tabular}
\end{frame}
\begin{frame}{JK-Flip-Flop}
\end{frame}
\begin{frame}{D-Flip-Flop mit JK}
\end{frame}
\begin{frame}{T-Flip-Flop Tabelle}
	\centering
	\begin{tabular}[t]{c|c}
		\(T\)  & \(Q\) \\
		\hline
		0         & \\
		1         & \\
	\end{tabular}
\end{frame}
\begin{frame}{T-Flip-Flop}
\end{frame}

\section{VHDL Signale}
\begin{frame}{Signaldeklerationen}
	\centering
	\begin{tabular}[t]{c|ccccccccc}
		Signal     & A    & B    & C    & D    & E    & F    & G    & H    & I \\
		\hline
		Bits    &     &     &     &     &     & &     &     &  \\
	\end{tabular}
\end{frame}
\begin{frame}{Deklarationen}
	\begin{itemize}
		\item signal K : 
		\item signal L :
		\item signal M :
		\item signal N :
		\item signal O :
	\end{itemize}
\end{frame}
\begin{frame}{Zuweisungen}
	\begin{itemize}
		\item A <=
		\item B <=
		\item C <=
		\item D <=
		\item E <=
		\item F <= 
	\end{itemize}
\end{frame}
\end{document}
