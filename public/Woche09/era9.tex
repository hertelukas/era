%%
% This is an Overleaf template for presentations
% using the TUM Corporate Desing https://www.tum.de/cd
%
% For further details on how to use the template, take a look at our
% GitLab repository and browse through our test documents
% https://gitlab.lrz.de/latex4ei/tum-templates.
%
% The tumbeamer class is based on the beamer class.
% If you need further customization please consult the beamer class guide
% https://ctan.org/pkg/beamer.
% Additional class options are passed down to the base class.
%
% If you encounter any bugs or undesired behaviour, please raise an issue
% in our GitLab repository
% https://gitlab.lrz.de/latex4ei/tum-templates/issues
% and provide a description and minimal working example of your problem.
%%


\documentclass[
  german,            % define the document language (english, german)
  aspectratio=169,    % define the aspect ratio (169, 43)
  % handout=2on1,       % create handout with multiple slides (2on1, 4on1)
  % partpage=false,     % insert page at beginning of parts (true, false)
  sectionpage=false,   % insert page at beginning of sections (true, false)
]{tumbeamer}


% load additional packages
\usepackage{booktabs}
\usepackage{listings}


% presentation metadata
\title{Einführung in die Rechnerarchitektur}
\subtitle{Schaltnetze}
\author{Lukas Hertel}

\institute{\theChairName\\\theDepartmentName\\\theUniversityName}
\date[07.01.2022]{07. Januar 2022}

\footline{\insertauthor~|~\insertshorttitle~|~\insertshortdate}


% macro to configure the style of the presentation
\TUMbeamersetup{
  title page = TUM tower,         % style of the title page
  part page = TUM toc,            % style of part pages
  section page = TUM toc,         % style of section pages
  content page = TUM more space,  % style of normal content pages
  tower scale = 1.0,              % scaling factor of TUM tower (if used)
  headline = TUM threeliner,      % which variation of headline to use
  footline = TUM default,         % which variation of footline to use
  % configure on which pages headlines and footlines should be printed
  headline on = {title page},
  footline on = {every page, title page=false},
}

% available frame styles for title page, part page, and section page:
% TUM default, TUM tower, TUM centered,
% TUM blue default, TUM blue tower, TUM blue centered,
% TUM shaded default, TUM shaded tower, TUM shaded centered,
% TUM flags
%
% additional frame styles for part page and section page:
% TUM toc
%
% available frame styles for content pages:
% TUM default, TUM more space
%
% available headline options:
% TUM empty, TUM oneliner, TUM twoliner, TUM threeliner, TUM logothreeliner
%
% available footline options:
% TUM empty, TUM default, TUM infoline


\begin{document}

\maketitle

\section{Aufgabe 1}
\begin{frame}{Regeln der boolschen Algebra}

\end{frame}
\begin{frame}{Eindimensionale Schaltfunktionen}
\end{frame}
\begin{frame}{Schaltfunktionen mit zwei Variablen}
		\begin{center}\begin{tabular}{|cc|cccccccccccccccc|}
			\hline
			$a$ & $b$ & $f_0$ & $f_1$ & $f_2$ & $f_3$ & $f_4$ & $f_5$ & $f_6$ & $f_7$ & $f_8$ & $f_9$ & $f_{10}$ & $f_{11}$ & $f_{12}$ & $f_{13}$ & $f_{14}$ & $f_{15}$ \\
			\hline
			$0$ & $0$ & & & & & & & & & & & & & & & &
			\\			
			$0$ & $1$ & & & & & & & & & & & & & & & &
			\\
			$1$ & $0$ & & & & & & & & & & & & & & & &
			\\
			$1$ & $1$ & & & & & & & & & & & & & & & &
			\\
			\multicolumn{2}{|c|}{$Name$} & & & & & & & & & & & & & & & & 
			\\
			
			\hline
	\end{tabular}\end{center}
\end{frame}
\begin{frame}{Schaltfunktionen mit zwei Variablen}{Lösung}
	\begin{center}\begin{tabular}{|cc|cccccccccccccccc|}
			\hline
			$a$ & $b$ & $f_0$ & $f_1$ & $f_2$ & $f_3$ & $f_4$ & $f_5$ & $f_6$ & $f_7$ & $f_8$ & $f_9$ & $f_{10}$ & $f_{11}$ & $f_{12}$ & $f_{13}$ & $f_{14}$ & $f_{15}$ \\
			\hline
			$0$ & $0$ & $0$   & $0$   & $0$   & $0$   & $0$   & $0$   & $0$   & $0$   & $1$   & $1$   & $1$    & $1$    & $1$    & $1$    & $1$    & $1$    \\
			$0$ & $1$ & $0$   & $0$   & $0$   & $0$   & $1$   & $1$   & $1$   & $1$   & $0$   & $0$   & $0$    & $0$    & $1$    & $1$    & $1$    & $1$    \\
			$1$ & $0$ & $0$   & $0$   & $1$   & $1$   & $0$   & $0$   & $1$   & $1$   & $0$   & $0$   & $1$    & $1$    & $0$    & $0$    & $1$    & $1$    \\
			$1$ & $1$ & $0$   & $1$   & $0$   & $1$   & $0$   & $1$   & $0$   & $1$   & $0$   & $1$   & $0$    & $1$    & $0$    & $1$    & $0$    & $1$    \\
			& & & & & & & & & & & & & & & & &
			\\			
			& & & & & & & & & & & & & & & & &
			\\

			\hline
	\end{tabular}\end{center}
	
	
\end{frame}
\begin{frame}{Logikfunktion}{Wahrheitstabelle}
\begin{tabular}[t]{ccc|r}
	a & b & c & r \\
	\hline
	0 & 0 & 0 &  \\
	0 & 0 & 1 &  \\
	0 & 1 & 0 &  \\
	0 & 1 & 1 &  \\
	1 & 0 & 0 &  \\
	1 & 0 & 1 &  \\
	1 & 1 & 0 &  \\
	1 & 1 & 1 &  \\
\end{tabular}
\end{frame}
\begin{frame}{KNF \& DNF}
	\begin{itemize}
		\item KNF
		\begin{itemize}
			\item Verundete Oders
			\item Auswahl der 0en im Resultat, negiert falls 1
		\end{itemize}
		\item DNF
		\begin{itemize}
			\item Veroderte Unds
			\item Auswahl der 1en im Resultat, negiert falls 0
			\item Intuitiver
		\end{itemize}
	\end{itemize}
\end{frame}
\begin{frame}{KNF \& DNF}{Lösung}
\end{frame}
\begin{frame}{Noch mehr KNF \& DNF}
\end{frame}

\end{document}
