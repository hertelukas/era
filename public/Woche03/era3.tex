%%
% This is an Overleaf template for presentations
% using the TUM Corporate Desing https://www.tum.de/cd
%
% For further details on how to use the template, take a look at our
% GitLab repository and browse through our test documents
% https://gitlab.lrz.de/latex4ei/tum-templates.
%
% The tumbeamer class is based on the beamer class.
% If you need further customization please consult the beamer class guide
% https://ctan.org/pkg/beamer.
% Additional class options are passed down to the base class.
%
% If you encounter any bugs or undesired behaviour, please raise an issue
% in our GitLab repository
% https://gitlab.lrz.de/latex4ei/tum-templates/issues
% and provide a description and minimal working example of your problem.
%%


\documentclass[
  german,            % define the document language (english, german)
  aspectratio=169,    % define the aspect ratio (169, 43)
  % handout=2on1,       % create handout with multiple slides (2on1, 4on1)
  % partpage=false,     % insert page at beginning of parts (true, false)
  sectionpage=false,   % insert page at beginning of sections (true, false)
]{tumbeamer}


% load additional packages
\usepackage{booktabs}
\usepackage{listings}

% presentation metadata
\title{Einführung in die Rechnerarchitektur}
\subtitle{Speicherwerk und Stack}
\author{Lukas Hertel}

\institute{\theChairName\\\theDepartmentName\\\theUniversityName}
\date[08.11.2021]{08. November 2021}

\footline{\insertauthor~|~\insertshorttitle~|~\insertshortdate}


% macro to configure the style of the presentation
\TUMbeamersetup{
  title page = TUM tower,         % style of the title page
  part page = TUM toc,            % style of part pages
  section page = TUM toc,         % style of section pages
  content page = TUM more space,  % style of normal content pages
  tower scale = 1.0,              % scaling factor of TUM tower (if used)
  headline = TUM threeliner,      % which variation of headline to use
  footline = TUM default,         % which variation of footline to use
  % configure on which pages headlines and footlines should be printed
  headline on = {title page},
  footline on = {every page, title page=false},
}

% available frame styles for title page, part page, and section page:
% TUM default, TUM tower, TUM centered,
% TUM blue default, TUM blue tower, TUM blue centered,
% TUM shaded default, TUM shaded tower, TUM shaded centered,
% TUM flags
%
% additional frame styles for part page and section page:
% TUM toc
%
% available frame styles for content pages:
% TUM default, TUM more space
%
% available headline options:
% TUM empty, TUM oneliner, TUM twoliner, TUM threeliner, TUM logothreeliner
%
% available footline options:
% TUM empty, TUM default, TUM infoline


\begin{document}

\maketitle

\section{Hausaufgabe}
\begin{frame}{Hausaufgabe}

\end{frame}
\section{Aufgabe 1}
\begin{frame}{Speicherorganisation}{Speicherpyramide}
\end{frame}
\begin{frame}{Speicherorganisation}{Struktur}
	\begin{itemize}
		\item Wie ist der Speicher strukturiert?
		\item Wie ist der Speicher mit dem Leit- und Rechenwerk verbunden?
		\item Was muss man beim Datenzugriff beachten, wenn das gesuchte Datenwort im Cache
		steht?
	\end{itemize}
\end{frame}
\begin{frame}{Speicherorganisation}{Little- und Big-Endian}
	\begin{itemize}
		\item Was ist der Unterschied zwischen Little- und Big-Endian?
		\item Ablage des Wortes 0x76543210 im Speicher
	\end{itemize}
\vspace{1cm}
\begin{minipage}[t]{0.45\textwidth}
	Little Endian
	\begin{tabular}{r|c|}
		N+3 & \hspace{1cm} \\
		N+2 & \\
		N+1 & \\
		Adresse N & \\
	\end{tabular}
\end{minipage}
\begin{minipage}[t]{0.45\textwidth}
	Big Endian
	\begin{tabular}{r|c|}
		N+3 & \hspace{1cm} \\
		N+2 & \\
		N+1 & \\
		Adresse N & \\
	\end{tabular}
\end{minipage}
\end{frame}
\begin{frame}{Speicherorganisation}{.bss und .data}
	\begin{itemize}
		\item \texttt{.bss} für statische Variablen die deklariert wurden, aber \textbf{nicht initialisiert}
		\item \texttt{.data} für statische Variablen, die \textbf{initialisiert} wurden
	\end{itemize}
\end{frame}
\section{Aufgabe 2}
\begin{frame}{Speicherzugriffe}
	\begin{itemize}
		\item \href{https://css.csail.mit.edu/6.858/2015/readings/i386.pdf}{i386 manual}
	\end{itemize}
\end{frame}
\begin{frame}{Speicherzugriffe}
	\begin{itemize}
		\item \href{https://css.csail.mit.edu/6.858/2015/readings/i386.pdf}{i386 manual}
		\item Auf der x86 Architektur kann immer nur ein Operand ein Speicherzugriff sein
	\end{itemize}
\end{frame}
\section{Aufgabe 3}
\begin{frame}{Zeichenketten \& Arrays}{Adressierungsarten}
\end{frame}
\end{document}
