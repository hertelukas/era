%%
% This is an Overleaf template for presentations
% using the TUM Corporate Desing https://www.tum.de/cd
%
% For further details on how to use the template, take a look at our
% GitLab repository and browse through our test documents
% https://gitlab.lrz.de/latex4ei/tum-templates.
%
% The tumbeamer class is based on the beamer class.
% If you need further customization please consult the beamer class guide
% https://ctan.org/pkg/beamer.
% Additional class options are passed down to the base class.
%
% If you encounter any bugs or undesired behaviour, please raise an issue
% in our GitLab repository
% https://gitlab.lrz.de/latex4ei/tum-templates/issues
% and provide a description and minimal working example of your problem.
%%


\documentclass[
  german,            % define the document language (english, german)
  aspectratio=169,    % define the aspect ratio (169, 43)
  % handout=2on1,       % create handout with multiple slides (2on1, 4on1)
  % partpage=false,     % insert page at beginning of parts (true, false)
  sectionpage=false,   % insert page at beginning of sections (true, false)
]{tumbeamer}


% load additional packages
\usepackage{booktabs}
\usepackage{listings}

% presentation metadata
\title{Einführung in die Rechnerarchitektur}
\subtitle{Flags und bedingte Sprünge}
\author{Lukas Hertel}

\institute{\theChairName\\\theDepartmentName\\\theUniversityName}
\date[22.11.2021]{22. November 2021}

\footline{\insertauthor~|~\insertshorttitle~|~\insertshortdate}


% macro to configure the style of the presentation
\TUMbeamersetup{
  title page = TUM tower,         % style of the title page
  part page = TUM toc,            % style of part pages
  section page = TUM toc,         % style of section pages
  content page = TUM more space,  % style of normal content pages
  tower scale = 1.0,              % scaling factor of TUM tower (if used)
  headline = TUM threeliner,      % which variation of headline to use
  footline = TUM default,         % which variation of footline to use
  % configure on which pages headlines and footlines should be printed
  headline on = {title page},
  footline on = {every page, title page=false},
}

% available frame styles for title page, part page, and section page:
% TUM default, TUM tower, TUM centered,
% TUM blue default, TUM blue tower, TUM blue centered,
% TUM shaded default, TUM shaded tower, TUM shaded centered,
% TUM flags
%
% additional frame styles for part page and section page:
% TUM toc
%
% available frame styles for content pages:
% TUM default, TUM more space
%
% available headline options:
% TUM empty, TUM oneliner, TUM twoliner, TUM threeliner, TUM logothreeliner
%
% available footline options:
% TUM empty, TUM default, TUM infoline


\begin{document}

\maketitle

\section{Hausaufgabe}
\begin{frame}{Hausaufgabe}
\end{frame}

\section{Aufgabe 1 - Flags}
\begin{frame}{Statusregister}
\begin{table}[]
	\begin{tabular}{llll}
		\hline
		\textbf{Kurz} & \textbf{Name} & \textbf{Beschreibung} & \textbf{Bei \texttt{CMP EAX, EBX} 1 falls}                                \\ \hline
		CF                 & Carry Flag    & Übertrag              & EAX \textless \space EBX (ohne Vorzeichen)              \\
		ZF                 & Zero Flag     & Ergebnis ist null     & EAX = EBX (Vorzeichen egal)                      \\
		SF                 & Sign Flag     & Vorzeichen            & Register negativ                                 \\
		OF                 & Overflow Flag & Überlauf              & EAX - EBX nicht repräsentierbar (mit Vorzeichen) \\ \hline
	\end{tabular}
\end{table}
\end{frame}
\begin{frame}{Aufgabe 1a}
			\begin{center}
		\begin{tabular}{r l c c c c}
			& Befehlsfolge & Carry & Overflow & Sign & Zero\\ \hline
			1: & - & 0 & 1 & 0 & 1\\
			2: & MOV EAX, 300 & \underline{~~~~} & \underline{~~~~} & \underline{~~~~} & \underline{~~~~}\\
			3: & SUB AL, 100 & \underline{~~~~} & \underline{~~~~} & \underline{~~~~} & \underline{~~~~}\\
			4: & MOV AX, 300 & \underline{~~~~} & \underline{~~~~} & \underline{~~~~} & \underline{~~~~}\\
			5: & CMP AH, 200 & \underline{~~~~} & \underline{~~~~} & \underline{~~~~} & \underline{~~~~}\\
			6: & ADD AL, 100 & \underline{~~~~} & \underline{~~~~} & \underline{~~~~} & \underline{~~~~}
		\end{tabular}
	\end{center}
\end{frame}
\begin{frame}{Aufgabe 1b}
		\begin{center}
	\begin{tabular}{r l c c c c}
		& Befehlsfolge & Carry & Overflow & Sign & Zero\\ \hline
		1: & - & 0 & 0 & 0 & 1\\
		2: & MOV EAX, -120 & \underline{~~~~} & \underline{~~~~} & \underline{~~~~} & \underline{~~~~}\\
		3: & PUSH AX & \underline{~~~~} & \underline{~~~~} & \underline{~~~~} & \underline{~~~~}\\
		4: & CMP AL, 0 & \underline{~~~~} & \underline{~~~~} & \underline{~~~~} & \underline{~~~~}\\
		5: & ADD AL, 140 & \underline{~~~~} & \underline{~~~~} & \underline{~~~~} & \underline{~~~~}
	\end{tabular}
\end{center}
\end{frame}
\section{Aufgabe 2}
\begin{frame}{Bedingte Sprünge}{Möglichkeiten um zur \texttt{marke1} zu springen}
\end{frame}
\begin{frame}[fragile]{Von Java zu Assembler}
Wo setzen wir die Marken?
\newline
	\lstset{language=C}
	\begin{lstlisting}
		int ebx;
		ebx = 50;
		while (ebx <= 60) {
			fkt(ebx);
			ebx = ebx + 1;
		}
	\end{lstlisting}
\end{frame}
\section{Aufgabe 4 - Shift}
\begin{frame}{Multiplikation mit Shifts}
	\begin{itemize}
		\item \texttt{SHL/SAL/SHR n}
		\begin{itemize}
			\item Shift um \texttt{n} Stellen
			\item Wird mit 0en aufgefüllt
		\end{itemize}
		\item \texttt{SAR n}
		\begin{itemize}
			\item Shift um \texttt{n} Stellen
			\item Wird mit 1en aufgefüllt (Wieso?)
		\end{itemize}
		\item Was passiert bei einem Shift um $n$ Stellen?
	\end{itemize}
\end{frame}
\section{Aufgabe 3 - Bitoperationen}
\begin{frame}{Verscheibung von Bits}

\end{frame}
\begin{frame}{Einfügen von Bits in ein anderes Register}
\end{frame}
\begin{frame}{Bits prüfen auf Bedingung}
\end{frame}
\begin{frame}{Schnelle Division ohne \texttt{div}}
\end{frame}
\begin{frame}{Modulo}
\end{frame}
\begin{frame}{Nullsetzen von bit $n$}
\end{frame}
\end{document}
