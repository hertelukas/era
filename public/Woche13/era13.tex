% LTex: language=de-DE
%%
% This is an Overleaf template for presentations
% using the TUM Corporate Desing https://www.tum.de/cd
%
% For further details on how to use the template, take a look at our
% GitLab repository and browse through our test documents
% https://gitlab.lrz.de/latex4ei/tum-templates.
%
% The tumbeamer class is based on the beamer class.
% If you need further customization please consult the beamer class guide
% https://ctan.org/pkg/beamer.
% Additional class options are passed down to the base class.
%
% If you encounter any bugs or undesired behaviour, please raise an issue
% in our GitLab repository
% https://gitlab.lrz.de/latex4ei/tum-templates/issues
% and provide a description and minimal working example of your problem.
%%


\documentclass[
  german,            % define the document language (english, german)
  aspectratio=169,    % define the aspect ratio (169, 43)
  % handout=2on1,       % create handout with multiple slides (2on1, 4on1)
  % partpage=false,     % insert page at beginning of parts (true, false)
  sectionpage=false,   % insert page at beginning of sections (true, false)
]{tumbeamer}


% load additional packages
\usepackage{booktabs}
\usepackage{listings}

% presentation metadata
\title{Einführung in die Rechnerarchitektur}
\subtitle{Parallelisierung und Caches}
\author{Lukas Hertel}

\institute{\theChairName\\\theDepartmentName\\\theUniversityName}
\date[31.01.2022]{31. Januar 2022}

\footline{\insertauthor~|~\insertshorttitle~|~\insertshortdate}


% macro to configure the style of the presentation
\TUMbeamersetup{
  title page = TUM tower,         % style of the title page
  part page = TUM toc,            % style of part pages
  section page = TUM toc,         % style of section pages
  content page = TUM more space,  % style of normal content pages
  tower scale = 1.0,              % scaling factor of TUM tower (if used)
  headline = TUM threeliner,      % which variation of headline to use
  footline = TUM default,         % which variation of footline to use
  % configure on which pages headlines and footlines should be printed
  headline on = {title page},
  footline on = {every page, title page=false},
}

% available frame styles for title page, part page, and section page:
% TUM default, TUM tower, TUM centered,
% TUM blue default, TUM blue tower, TUM blue centered,
% TUM shaded default, TUM shaded tower, TUM shaded centered,
% TUM flags
%
% additional frame styles for part page and section page:
% TUM toc
%
% available frame styles for content pages:
% TUM default, TUM more space
%
% available headline options:
% TUM empty, TUM oneliner, TUM twoliner, TUM threeliner, TUM logothreeliner
%
% available footline options:
% TUM empty, TUM default, TUM infoline


\begin{document}

\maketitle

\section{Wiederholung}
\begin{frame}{Wiederholung}{Roofline Modell}
  \begin{itemize}
    \item Applikation limitiert durch
    \begin{itemize}
      \item Speichervolumen oder
      \item Rechenleistung
    \end{itemize}
    \item x-Achse: Arithmetische Intensität
    \item y-Achse: Rechenleistung
    \item Schätzt Top-Performance von Programmen ab
  \end{itemize}
\end{frame}
\begin{frame}{Wiederholung}{Parallele Leistungs Modelle}
\begin{itemize}
  \item Amdahl's Law
  \begin{itemize}
    \item Programme mit konstanter Problemgröße
    \item Sequentieller Anteil $t_s$ und parallelisierbarer Anteil $t_p$
    \item Normalisiert durch Gesamtanteil $s = t_s / (t_s + t_p)$ (Analog für $p$)
    \item \textbf{Speedup}: $S_{Amdahl}(n) = \frac{1}{s+\frac{p}{n}}$ mit $n$ Prozessoren
    \item Speedup ist beschränkt
  \end{itemize}
  \item Gustafson's Law 
  \begin{itemize}
    \item Problemgröße steigt mit Anzahl Prozessoren 
    \item \textbf{Speedup}: $S_{Gustafson}(n)=s+n*p$
    \item Speedup ist unbeschränkt
  \end{itemize}
\end{itemize}
\end{frame}
\section{Aufgabe 1}
\begin{frame}{Performance von Computern}{FLOP/s: Floating Point Operations per second}
  \begin{itemize}
    \item Anzahl Kerne:
    \item Bits, die berechnet werden, pro Kern:
    \item Gleichzeitige Fließkommaoperationen pro Kern:
    \item FLOP/s:
  \end{itemize}
\end{frame}
\begin{frame}{Performance von Computern}{Speicherbandbreite}
\end{frame}
\begin{frame}{Performance von Computern}{Roofline Modell}
\end{frame}
\begin{frame}{Maximale Performance}
\end{frame}
\begin{frame}[fragile]{Speedup}{Elemente zählen}
\begin{lstlisting}[language=C, mathescape]
int count(node *head) {
    node *cursor = head; $\textcolor{gray}{\texttt{//100ns}}$
    int c = 0; $\textcolor{gray}{\texttt{//1ns}}$
    //Solange cursor auf einen Knoten zeigt
    while(cursor != nullptr) { $\textcolor{gray}{\texttt{//1ns}}$
        c++; $\textcolor{gray}{\texttt{//1ns}}$
        // Setzt Cursor auf den eigenen Listennachfolger
        cursor = cursor->next; $\textcolor{gray}{\texttt{//100ns}}$
    }
    return c; $\textcolor{gray}{\texttt{//1ns}}$
}
\end{lstlisting}
\end{frame}
\begin{frame}[fragile]{Speedup}{Maximum}
\begin{lstlisting}[language=C, mathescape]
  int x[100];
  ...
  int find_max() {
          int max_val = x[0];  $\textcolor{gray}{\texttt{//100ns}}$
          for(int i=0; i<100; i++) {  $\textcolor{gray}{\texttt{//3ns}}$
                  max_val = max(x[i], max_val);  $\textcolor{gray}{\texttt{//17ns}}$
          }
          return max_val;  $\textcolor{gray}{\texttt{//1ns}}$
  }
  \end{lstlisting}
\end{frame}
\end{document}
