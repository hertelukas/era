%%
% This is an Overleaf template for presentations
% using the TUM Corporate Desing https://www.tum.de/cd
%
% For further details on how to use the template, take a look at our
% GitLab repository and browse through our test documents
% https://gitlab.lrz.de/latex4ei/tum-templates.
%
% The tumbeamer class is based on the beamer class.
% If you need further customization please consult the beamer class guide
% https://ctan.org/pkg/beamer.
% Additional class options are passed down to the base class.
%
% If you encounter any bugs or undesired behaviour, please raise an issue
% in our GitLab repository
% https://gitlab.lrz.de/latex4ei/tum-templates/issues
% and provide a description and minimal working example of your problem.
%%


\documentclass[
  german,            % define the document language (english, german)
  aspectratio=169,    % define the aspect ratio (169, 43)
  % handout=2on1,       % create handout with multiple slides (2on1, 4on1)
  % partpage=false,     % insert page at beginning of parts (true, false)
  sectionpage=false,   % insert page at beginning of sections (true, false)
]{tumbeamer}


% load additional packages
\usepackage{booktabs}
\usepackage{listings}


% presentation metadata
\title{Einführung in die Rechnerarchitektur}
\subtitle{Assembler}
\author{Lukas Hertel}

\institute{\theChairName\\\theDepartmentName\\\theUniversityName}
\date[01.11.2021]{11. November 2021}

\footline{\insertauthor~|~\insertshorttitle~|~\insertshortdate}


% macro to configure the style of the presentation
\TUMbeamersetup{
  title page = TUM tower,         % style of the title page
  part page = TUM toc,            % style of part pages
  section page = TUM toc,         % style of section pages
  content page = TUM more space,  % style of normal content pages
  tower scale = 1.0,              % scaling factor of TUM tower (if used)
  headline = TUM threeliner,      % which variation of headline to use
  footline = TUM default,         % which variation of footline to use
  % configure on which pages headlines and footlines should be printed
  headline on = {title page},
  footline on = {every page, title page=false},
}

% available frame styles for title page, part page, and section page:
% TUM default, TUM tower, TUM centered,
% TUM blue default, TUM blue tower, TUM blue centered,
% TUM shaded default, TUM shaded tower, TUM shaded centered,
% TUM flags
%
% additional frame styles for part page and section page:
% TUM toc
%
% available frame styles for content pages:
% TUM default, TUM more space
%
% available headline options:
% TUM empty, TUM oneliner, TUM twoliner, TUM threeliner, TUM logothreeliner
%
% available footline options:
% TUM empty, TUM default, TUM infoline


\begin{document}

\maketitle

\section{Hausaufgabe}
\begin{frame}{Hausaufgabe}
	\begin{itemize}
		\item $(00011010)_2 + (00100011)_2 = (?)_2$
		\item $(00110110)_2 - (00101010)_2 = (?)_2$
		\item $(00011110)_2 \cdot (00000101)_2 = (?)_2$
	\end{itemize}
\end{frame}

\section{Einführung}
\begin{frame}{Themen}
	\begin{itemize}
		\item Tutorblatt
		\begin{itemize}
			\item Registerbenutzung und Konstanten (MOV)
			\item MOV in ASM vs. Variablen in Hochsprachen
			\item Arithmetische Begriffe (ADD, SUB, NEG, MUL, DIV, MOD)
			\item Assembler zu Maschinencode
		\end{itemize}
		\item Hausaufgabe
		\begin{itemize}
			\item Festkommarechnung
		\end{itemize}
	\end{itemize}
\end{frame}
\begin{frame}{NASM}{Netwide Assembler}
	\begin{itemize}
		\item Plattformunabhängiger Assembler für x86 CPU Architektur
		\item Kommandozeilenwerkzeug
		\item Manuelles Kompilieren von Code, Syntax
		\item SASM basiert auf NASM
	\end{itemize}
\end{frame}
\begin{frame}{SASM}{SimpleASM}
	\begin{itemize}
		\item Grafische Entwicklungsumgebung für ASM
		\item Kompilieren, Debugging, I/0
		\item Benötigt Assemblierer und C-Compiler (z.B. nasm \& gcc)
		\item \href{https://dman95.github.io/SASM}{Link} für die Installation
	\end{itemize}
\end{frame}
\section{Aufgabe 1}
\begin{frame}{Von-Neumann-Architektur}
	
\end{frame}
\begin{frame}[fragile]{Register}{Teilregister von EAX}
\begin{table}[]
	\begin{tabular}{llllllll}
		0000 & 0000 & 0000 & \multicolumn{1}{l|}{0000} & 0000 & 0000 & 0000 & 0000 \\
		&  &  &  &  &  &  &  \\
		&  &  & \multicolumn{1}{l|}{} &  &  &  &  \\
		&  &  & \multicolumn{1}{l|}{} &  & \multicolumn{1}{l|}{} &  & 
	\end{tabular}
\end{table}
\end{frame}
\section{Aufgabe 2}
\begin{frame}{Aufgabe 2}{Unterschied MOV vs. Variablenzuweisung}
\end{frame}
\section{Aufgabe 3}
\begin{frame}{Aufgabe 3}{Arithmetische Befehle}
	\begin{itemize}
		\item ADD A, B => $A = A + B$
		\begin{itemize}
			\item MOV, SUB, AND, OR
		\end{itemize}
		\item NEG A => $A = -A$
		\begin{itemize}
			\item INC, DEC
		\end{itemize}
		\item MUL A -> $EDX:EAX = EAX * A$
		\item DIV A -> $EDX:EAX = EDX:EAX / A$
		\item IMUL (Signed multiply)
		\begin{itemize}
			\item A -> $EDX:EAX = EAX * A$
			\item A, B -> $A = A * B$
			\item A, B, C -> $A = B * C$
		\end{itemize}
	\end{itemize}
\end{frame}
\end{document}
