%%
% This is an Overleaf template for presentations
% using the TUM Corporate Desing https://www.tum.de/cd
%
% For further details on how to use the template, take a look at our
% GitLab repository and browse through our test documents
% https://gitlab.lrz.de/latex4ei/tum-templates.
%
% The tumbeamer class is based on the beamer class.
% If you need further customization please consult the beamer class guide
% https://ctan.org/pkg/beamer.
% Additional class options are passed down to the base class.
%
% If you encounter any bugs or undesired behaviour, please raise an issue
% in our GitLab repository
% https://gitlab.lrz.de/latex4ei/tum-templates/issues
% and provide a description and minimal working example of your problem.
%%


\documentclass[
  german,            % define the document language (english, german)
  aspectratio=169,    % define the aspect ratio (169, 43)
  % handout=2on1,       % create handout with multiple slides (2on1, 4on1)
  % partpage=false,     % insert page at beginning of parts (true, false)
  sectionpage=false,   % insert page at beginning of sections (true, false)
]{tumbeamer}


% load additional packages
\usepackage{booktabs}
\usepackage{listings}

% presentation metadata
\title{Einführung in die Rechnerarchitektur}
\subtitle{Zahlensysteme}
\author{Lukas Hertel}

\institute{\theChairName\\\theDepartmentName\\\theUniversityName}
\date[25.10.2021]{25. Oktober 2021}

\footline{\insertauthor~|~\insertshorttitle~|~\insertshortdate}


% macro to configure the style of the presentation
\TUMbeamersetup{
  title page = TUM tower,         % style of the title page
  part page = TUM toc,            % style of part pages
  section page = TUM toc,         % style of section pages
  content page = TUM more space,  % style of normal content pages
  tower scale = 1.0,              % scaling factor of TUM tower (if used)
  headline = TUM threeliner,      % which variation of headline to use
  footline = TUM default,         % which variation of footline to use
  % configure on which pages headlines and footlines should be printed
  headline on = {title page},
  footline on = {every page, title page=false},
}

% available frame styles for title page, part page, and section page:
% TUM default, TUM tower, TUM centered,
% TUM blue default, TUM blue tower, TUM blue centered,
% TUM shaded default, TUM shaded tower, TUM shaded centered,
% TUM flags
%
% additional frame styles for part page and section page:
% TUM toc
%
% available frame styles for content pages:
% TUM default, TUM more space
%
% available headline options:
% TUM empty, TUM oneliner, TUM twoliner, TUM threeliner, TUM logothreeliner
%
% available footline options:
% TUM empty, TUM default, TUM infoline


\begin{document}

\maketitle

\section{Vorstellung}
\begin{frame}{Vorstellung}
  \begin{itemize}
    \item 3. Semester Informatik
    \item Kontaktmöglichkeiten
    \begin{itemize}
        \item lukas.hertel@tum.de
        \item Zulip
    \end{itemize}
    \item \href{https://era.lukas-hertel.de}{Internetseite}
    \begin{itemize}
    	\item Folien
    	\item Hilfsunterlagen
    \end{itemize}
	\item Fragen?
  \end{itemize}
\end{frame}


\section{Datendarstellung}
\begin{frame}{Zahlensysteme}
	\begin{itemize}
		\item Binär
		\begin{itemize}
			\item \texttt{0b1010 0011 1101}
			\item Darstellung in den meisten Computern
		\end{itemize}
		\item Dezimal
		\begin{itemize}
			\item \texttt{2621}
		\end{itemize}
		\item Hexadezimal
		\begin{itemize}
			\item \texttt{0xA3D}
			\item Eine Ziffer stellt 4 Bits dar
			\item Nützlich um binäre Zahlen leserlicher darzustellen
		\end{itemize}
	\end{itemize}
\end{frame}
\begin{frame}{Zahlensysteme}{Binäre Darstellung}
Was ergbit \texttt{0x73 - 0b0010 0101}? (Binär berechnen)
\end{frame}
\begin{frame}{Zahlensysteme}{\texttt{0x73 - 0b0010 0101}}
\begin{itemize}
    \item Beides in einheitliche Zahlendarstellung konvertieren
    \item Zweierkomplement bilden
    \item \texttt{0x73 + (- 0b0010 0101)} rechnen 
\end{itemize}
\end{frame}
\begin{frame}[fragile]{Zahlensysteme}{\texttt{0x73 - 0b0010 0101}}
\begin{itemize}
    \item \texttt{0x73 => 0x0111 0011}
    \item \texttt{-0b0010 0101 => 0b1101 1010 + 0b1 => 0b11011011}
\end{itemize}
\begin{lstlisting}
  0 1 1 1  0 0 1 1 (115)
+ 1 1 0 1  1 0 1 1 (-37)
------------------
1 0 1 0 0  1 1 1 0  (78)
\end{lstlisting}
\end{frame}

\section{Übungen}
\begin{frame}{Übungen}
\end{frame}
\end{document}
