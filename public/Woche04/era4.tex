%%
% This is an Overleaf template for presentations
% using the TUM Corporate Desing https://www.tum.de/cd
%
% For further details on how to use the template, take a look at our
% GitLab repository and browse through our test documents
% https://gitlab.lrz.de/latex4ei/tum-templates.
%
% The tumbeamer class is based on the beamer class.
% If you need further customization please consult the beamer class guide
% https://ctan.org/pkg/beamer.
% Additional class options are passed down to the base class.
%
% If you encounter any bugs or undesired behaviour, please raise an issue
% in our GitLab repository
% https://gitlab.lrz.de/latex4ei/tum-templates/issues
% and provide a description and minimal working example of your problem.
%%


\documentclass[
  german,            % define the document language (english, german)
  aspectratio=169,    % define the aspect ratio (169, 43)
  % handout=2on1,       % create handout with multiple slides (2on1, 4on1)
  % partpage=false,     % insert page at beginning of parts (true, false)
  sectionpage=false,   % insert page at beginning of sections (true, false)
]{tumbeamer}


% load additional packages
\usepackage{booktabs}
\usepackage{listings}

% presentation metadata
\title{Einführung in die Rechnerarchitektur}
\subtitle{}
\author{Lukas Hertel}

\institute{\theChairName\\\theDepartmentName\\\theUniversityName}
\date[15.11.2021]{15. November 2021}

\footline{\insertauthor~|~\insertshorttitle~|~\insertshortdate}


% macro to configure the style of the presentation
\TUMbeamersetup{
  title page = TUM tower,         % style of the title page
  part page = TUM toc,            % style of part pages
  section page = TUM toc,         % style of section pages
  content page = TUM more space,  % style of normal content pages
  tower scale = 1.0,              % scaling factor of TUM tower (if used)
  headline = TUM threeliner,      % which variation of headline to use
  footline = TUM default,         % which variation of footline to use
  % configure on which pages headlines and footlines should be printed
  headline on = {title page},
  footline on = {every page, title page=false},
}

% available frame styles for title page, part page, and section page:
% TUM default, TUM tower, TUM centered,
% TUM blue default, TUM blue tower, TUM blue centered,
% TUM shaded default, TUM shaded tower, TUM shaded centered,
% TUM flags
%
% additional frame styles for part page and section page:
% TUM toc
%
% available frame styles for content pages:
% TUM default, TUM more space
%
% available headline options:
% TUM empty, TUM oneliner, TUM twoliner, TUM threeliner, TUM logothreeliner
%
% available footline options:
% TUM empty, TUM default, TUM infoline


\begin{document}

\maketitle

\section{Hausaufgabe}
\begin{frame}{Hausaufgabe}
	
\end{frame}
\section{Adressierung}
\begin{frame}{Adressierung bei RISC und CISC}
    \begin{itemize}
        \item Wie kann man auf Adressen zugreifen bei x86?
        \item Ist das nötig? Wie machen das andere Plattformen?
        \item Beispiele für RISC?
    \end{itemize}
\end{frame}
\begin{frame}{Adressierung}{Zusammenfassung}
    \begin{itemize}
        \item \texttt{[BASE + (INDEX * SCALE) + DISPLACEMENT]}
        \begin{itemize}
            \item \texttt{SCALE} kann 1, 2 oder 4 sein
            \item \texttt{DISPLACEMENT} kann 1 byte oder 4 byte gross sein
            \item \texttt{BASE} und \texttt{INDEX} Register werden nicht verändert
        \end{itemize}
    \end{itemize}
\end{frame}
\section{Aufgabe 1}
\begin{frame}{Wiederholung zu Adressierungsformen}{\texttt[MOV EBX, [EAX + ECX * 4 + 12]}
    \begin{itemize}
        \item Darstellung in RISC (Nur \texttt{BASE})
        \item Kein \texttt{MUL} (da ineffizient)
        \item Nur gleiche Register dürfen sich verändern wie im Original
        \item Vorteile/Nachteile
    \end{itemize}
\end{frame}
\section{Aufgabe 2}
\begin{frame}[fragile]{Klassenstrukturen von Hochsprachen in Assembler}
    \begin{semiverbatim}
        class Struktur1{
        public:
            int wert;
            int feld1[16];
        };

        class Struktur2{
        public:
            Struktur1* feld2[8];
            int info;
        };
    \end{semiverbatim}                
\end{frame}
\section{Aufgabe 3}
\begin{frame}{Programmsprünge}
\end{frame}
\section{Aufgabe 4}
\begin{frame}{Unterprogrammaufruf}
    \begin{itemize}
        \item Unterschied Sprung \& Unterprogramm-Aufruf
        \item Stack bei Unterprogrammaufruf
    \end{itemize}
\end{frame}
\section{Aufgabe 5}
\begin{frame}{Möglichkeiten für einen Unterprogrammaufruf mit Parametern}
    \begin{itemize}
        \item Register
        \item Speicherstellen
        \item Stack
        \item Vorteile/Nachteile?
     \end{itemize}
\end{frame}
\end{document}
